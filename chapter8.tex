\chapter{Conclusion}
In Chapter 1 we looked at the benefits of multicasting and what kind security
threat the use of it entails in SMEs. The main framework of thinking we emphasized
is ultimately the traffic is coming from source with real IP address. The whole
design challenge in this security issue is how we can do the mapping of the group
virtual address to the real unicast source address. The dynamic stateful multicast
paradigm works on the premise 'if one contacts the group they are also likely to
contact the source they are seeking'. This, we said, is the strongest assumption of
the thesis. But, it is also a realistic assumption.
In Chapter 3 and 4, we looked at the algorithms and data structures we modelled
and implemented this idea with. We model it as to provide with a good analytical
view the problems that are we are faced. The way it can be conceived is as a
security glue between SSM and ASM. It tries to feel the crack that can be opened
when one steps to or remains in ASM service model.
We emphasized the design challenges, observations and algorithms than our
particular implementation on the Linux Kernel. They are of greater importance if
anyone wants construct similar endeavour and as life cyle of kernel, as one facet of
technology, changes rapidly.
\appendix
\chapter{Module Code}
The compatibility source code for different versions of Kernel from the link:
http://github.com/bewket/adhoc-multicast/tree/master. It is taken from http://jengelh.medozas.de/ xtables-addons which seeks to provide that.
\section{README}
\begin{verbatim}
This module provides a filter for incoming multicast traffic by doing the f/g:
1. Snooping Subscibe (or join) messages and storing the client by its group hash.
2. We also snoop mulitcast source address (unicast type). 
This is the part where one sort of NEEDs
to pierce a hole for the multicast (which can be seen in the second last line)
 source-
its unicast address.
We store the client by the multicast source hash.
3. When multicast trafffic come we hash both group and source address and
Checking if the client exists in both hashes.
You need iptables >= 1.4.3 (x_tables module)
and kernel-source >= 2.6.17
Tools: socat-or netcat-, vlc
For testing purpose do the following (need to be root) :
# sh test.sh
Start vlc (www.videolan.org) stream video
on 238.234.234.1
Start extra vlc. Try to open the video on open network and entering the streaming
address. It should NOT open.
You will have to choose UDP in both cases.
Then do the following (after changing the 192.x.x.x to your ip)
echo "hi" | socat - udp-sendto:192.168.1.4:4444
and it should now open
You can download the code by using git. At the terminal type (this
$git clone git://github.com/bewket/adhoc-multicast.git
\end{verbatim}
\section{The filter Module}
\subsection{xt\_MCAST.h.}
\begin{verbatim}
#ifndef _XT_MCAST_TARGET_H
#define _XT_MCAST_TARGET_H
#include <linux/types.h>
struct xt_mcast_target_info {
uint32_t msrc_size;
uint32_t mgrp_size;
};
#endif /* XT_MCAST_TARGET_H */
\end{verbatim}
\subsection{xt\_MCAST.c.}
\begin{verbatim}
#define MSRC_SIZE 20 //src->cli_head_nodes
#define MGRP_SIZE 20 //grp->cli_head_nodes
#define MGRP_EXPIRE
120 //seconds
#include "compat_xtables.h"
#include <linux/skbuff.h>
#include <linux/module.h>
#include <linux/jiffies.h>
#include <net/ipv6.h>
#include <net/ip.h>
#include <linux/ip.h>
#include <linux/igmp.h>
#include <linux/list.h>
#include <linux/jhash.h>
#include <linux/netfilter.h>
#include <linux/netfilter/x_tables.h>
#include "xt_MCAST.h"
//you can compare addresses directly without htonl
//if (ipv6_addr_cmp(&iph->saddr, &new_addr) == 0)
static u_int32_t jhash_rnd;
struct xt_mcast {
struct hlist_node node;
__be32 ip;
unsigned long timeout;
//unsigned long
expire; //this is in jiffies (allowed per group)
};
struct mtable {
struct hlist_head members[0];
};
static struct mtable *msrc_hash;
//client list indexed by hash(src)->cli_ip
static struct mtable *mgrp_hash;
//client list indexed by hash(gid)->cli_ip
static unsigned int source_size = MSRC_SIZE;
static unsigned int group_size = MGRP_SIZE;
static unsigned int source_expire = 2 * 60 * 60;
//2-hours (has to
static DEFINE_MUTEX(mcast_mutex);
static DEFINE_SPINLOCK(mcast_lock);
static struct hlist_node *get_headnode(struct mtable *t, uint32_t h)
{
return t->members[h].first;
}
static unsigned int
init_new_entry(struct xt_mcast *entry, __be32 ip, uint32_t h,
struct mtable *t, uint16_t expire)
{
//we are not going to simply add it.
//go to the list if there is someone who has been expired
//start from the head downwards
//if the head is expired simple update the head with this
//we want to avoid as many memory alloc calls as possible
struct hlist_node *pos, *temp;
struct xt_mcast *temp_entry;
struct hlist_node *ref = get_headnode(t, h);
entry = NULL; //just in case
spin_lock_bh(&mcast_lock);
hlist_for_each(pos, &t->members[h]) {
temp_entry = hlist_entry(pos, struct xt_mcast, node);
if (temp_entry != NULL && time_after(jiffies, temp_entry->timeout)) {
//timeout
entry = temp_entry;
//the first time this we want to delete all but the first expired
//go until you find the first timeout and delete downwards.
pos = pos->next;
//don't delete pos- just advance it
while (pos != NULL && pos != ref) {
temp_entry = hlist_entry(pos, struct xt_mcast, node);
temp = pos;
pos = NULL; //No pter reshuffle here
pos = temp->next;
kfree(temp_entry);
}
break;
}
}
if (entry == NULL)
entry = kmalloc(sizeof(struct xt_mcast), GFP_ATOMIC);
else
printk("Updating entry\n");
entry->ip = ip;
entry->timeout = jiffies + expire * HZ;
hlist_add_head(&entry->node, &t->members[h]);
spin_unlock_bh(&mcast_lock);
return NF_ACCEPT;
}
static inline uint32_t mcast_hash(__be32 name, unsigned int size)
{
return jhash_1word(name, jhash_rnd) & (size - 1);
}
static struct hlist_node *lookup(__be32 hkey, __be32 ip, struct mtable *t)
{
struct xt_mcast *entry;
struct hlist_node *n = NULL;
if (t == NULL) {
printk("Table is NULL\n");
return NULL;
}
//use of the mutex should be appropriate here-Noisy Kernel unless new!!
spin_lock_bh(&mcast_lock);
hlist_for_each_entry(entry, n, &t->members[hkey], node) {
if (entry->ip == ip) {
spin_unlock_bh(&mcast_lock);
return n;
}
}
spin_unlock_bh(&mcast_lock);
return NULL;
}
static unsigned int igmp_report(__be32 cli, __be32 grp)
{
struct xt_mcast *entry = NULL, *head_entry;
//aka head
uint32_t h = mcast_hash(grp, group_size);
struct hlist_node *head_node = get_headnode(mgrp_hash, h);
struct hlist_node *entry_node;
printk("IGMP REPORT\n");
printk(NIPQUAD_FMT "->" NIPQUAD_FMT "\n", NIPQUAD(grp), NIPQUAD(cli));
if (head_node == NULL)
return init_new_entry(entry, cli, h, mgrp_hash, MGRP_EXPIRE);
//go fish
entry_node = lookup(h, cli, mgrp_hash);
//here there is grp alrdy but is the client in it.
if (entry_node == NULL) {
return init_new_entry(entry, cli, h, mgrp_hash, MGRP_EXPIRE);
}
spin_lock_bh(&mcast_lock);
//good the client is in the grp.
head_entry = hlist_entry(head_node, struct xt_mcast, node);
entry = hlist_entry(entry_node, struct xt_mcast, node);
entry->timeout = jiffies + MGRP_EXPIRE * HZ;
//we need to also update the same entry in the source list
if (entry->ip == head_entry->ip)
goto out;
//it the head. don't do anything
hlist_del(entry_node); //shuffle and fix
hlist_add_head(&entry->node, &mgrp_hash->members[h]);
//we want this node to be head so we need to del:
//NOTE: when deleting a grp if it the head pointer
out:spin_unlock_bh(&mcast_lock);
return NF_ACCEPT;
}
/* this probably will increase complexity and superflous somehow
* we can just ignore this and use
* our clean first and insert mechanism in fact this
removes
* temporal locality of our entries
*/
static unsigned igmp_leave(__be32 cli, __be32 grp)
{
uint32_t h = mcast_hash(grp, group_size);
struct hlist_node *entry_node;
struct xt_mcast *entry, *head_entry; //aka head
struct hlist_node *head_node = get_headnode(mgrp_hash, h);
/* entry is NULL => "I am leaving this grp"-
*yet there is no such grp or doesn't have me
* shouldn't be possible */
if (head_node == NULL)
return NF_ACCEPT;
entry_node = lookup(h, cli, mgrp_hash);
if (entry_node == NULL) //weird
return NF_ACCEPT;
entry = hlist_entry(entry_node, struct xt_mcast, node);
//some clients may just join and leave straight away after becoming last_seen
//Note: Also IGMPv1 don't issue leave
//The simple solution here is to make the head of the group last_seen.
//which often is true
head_entry = hlist_entry(head_node, struct xt_mcast, node);
entry = hlist_entry(entry_node, struct xt_mcast, node);
spin_lock_bh(&mcast_lock);
if (head_entry->ip == entry->ip && entry_node->next != NULL)
//promote the successor to be head
hlist_add_head(entry_node->next, &mgrp_hash->members[h]);
//else it is the last entry
hlist_del(entry_node);
spin_unlock_bh(&mcast_lock);
kfree(entry);
return NF_ACCEPT;
}
//Incoming traffic MCAST traffic
static unsigned int mcast_handler4(const struct iphdr *iph)
{
__be32 msrc_ip = iph->saddr;
__be32 mgrp_ip = iph->daddr;
uint32_t hs = mcast_hash(msrc_ip, source_size);
//h(msrc_ip)->cli_ip
uint32_t hg = mcast_hash(mgrp_ip, group_size);
//h(mgrp_ip)->cli_ip
struct xt_mcast *msrc2cli, *mgrp2cli;
struct hlist_node *n, *n2;
//check if there exists any client(s) that are mapped to msrc_ip
//while looking hash them and check if they have
spin_lock_bh(&mcast_lock);
hlist_for_each_entry(msrc2cli, n, &msrc_hash->members[hs], node) {
hlist_for_each_entry(mgrp2cli, n2, &mgrp_hash->members[hg], node) {
if (msrc2cli->ip == mgrp2cli->ip) { //success
spin_unlock_bh(&mcast_lock);
return NF_ACCEPT;
//remember we won't update anything here (
}
}
}
spin_unlock_bh(&mcast_lock);
return NF_DROP;
}
static unsigned
int igmp_handler(const struct iphdr *iph, const struct sk_buff *skb)
{
const struct igmphdr *igmph = igmp_hdr(skb);
const struct igmpv3_grec *rec3;
switch (igmph->type) {
case IGMP_HOST_MEMBERSHIP_REPORT:
case IGMPV2_HOST_MEMBERSHIP_REPORT:
return igmp_report(iph->saddr, iph->daddr);
case IGMPV3_HOST_MEMBERSHIP_REPORT:
//this special V2 equivalent case we can handle per rfc3376
if (ntohs(igmpv3_report_hdr(skb)->ngrec) == 1) {
//ONLY one grp
rec3 = igmpv3_report_hdr(skb)->grec;
if (ntohs(rec3->grec_nsrcs) == 0) //No src list
switch (rec3->grec_type) {
case IGMPV3_CHANGE_TO_EXCLUDE:
return igmp_report(iph->saddr, rec3->grec_mca);
case IGMPV3_CHANGE_TO_INCLUDE:
return igmp_leave(iph->saddr, rec3->grec_mca);
}
}
return NF_ACCEPT;
case IGMP_HOST_LEAVE_MESSAGE:
return igmp_leave(iph->saddr, igmph->group);
//from IGMP_V2 to delete a group (ehnancement on Time interval timer)
default:
return NF_ACCEPT;
}
return NF_ACCEPT;
}
static inline struct mtable *init_table(uint32_t nmembers)
{
uint32_t i;
struct mtable *t;
t = kzalloc(sizeof(*t) + nmembers * sizeof(t->members[0]), GFP_KERNEL);
for (i = 0; i < nmembers; i++)
INIT_HLIST_HEAD(&t->members[i]);
return t;
}
static void delete_list(struct hlist_head *head)
{
struct xt_mcast *entry;
struct hlist_node *pos, *temp;
pos = head->first;
while (pos != NULL) {
entry = hlist_entry(pos, struct xt_mcast, node);
temp = pos;
pos = NULL; //No pter reshuffle here
pos = temp->next;
kfree(entry);
}
}
static inline void delete_table(struct mtable *t, uint32_t nsize)
{
uint32_t i;
if (t == NULL)
return;
spin_lock_bh(&mcast_lock);
for (i = 0; i < nsize; i++)
delete_list(&t->members[i]);
spin_unlock_bh(&mcast_lock);
kfree(t);
}
/**OUTGOING traffic only thus client = ip_hrd(skb)->saddr;
* mcast_src_addr= ip_hdr(skb)->daddr;
*/
static unsigned int unicast_handler(const struct iphdr *iph)
{
__be32 mcli_ip = iph->saddr;
uint32_t hs = mcast_hash(iph->daddr, source_size);
struct hlist_node *mcli_node;
struct xt_mcast *mcli = NULL;
mcli_node = lookup(hs, mcli_ip, msrc_hash);
if (mcli_node == NULL)
return init_new_entry(mcli, mcli_ip, hs, msrc_hash, source_expire);
spin_lock_bh(&mcast_lock);
mcli = hlist_entry(mcli_node, struct xt_mcast, node);
mcli->timeout = jiffies + source_expire * HZ;
spin_unlock_bh(&mcast_lock);
return NF_ACCEPT;
}
static unsigned int mcast_tg4(const struct sk_buff **pskb,
const struct xt_target_param *par)
{
const struct sk_buff *skb = *pskb;
const struct xt_mcast_target_info *info = par->targinfo;
const struct iphdr *iph = ip_hdr(skb);
if (info) {
if (info->msrc_size > MGRP_SIZE)
source_size = info->msrc_size;
if (info->mgrp_size > MGRP_SIZE)
group_size = info->mgrp_size;
}
if (iph == NULL)
return NF_ACCEPT;
if (ipv4_is_multicast(iph->daddr))
//broadcast also mcast so: skb->pkt_type==PACKET_MULTICAST check??
{
if (ip_hdr(skb)->protocol == IPPROTO_UDP)
return mcast_handler4(iph);
//there maybe NF_DROPs
if (ip_hdr(skb)->protocol == IPPROTO_IGMP)
return igmp_handler(iph, skb);
//always NF_ACCEPT just for snooping
} else if (ip_hdr(skb)->protocol == IPPROTO_UDP) {
//the heartbeat protocol can be ICMP, or any other so long as we
// get its ip header to extract msrc.
return unicast_handler(iph);
//this needs to make fast construction of <s,c> pair
}
return NF_ACCEPT;
}
static struct xt_target mcast_tg_reg __read_mostly = {
.name = "MCAST",
.revision = 0,
.family = NFPROTO_IPV4,
.target = mcast_tg4,
.targetsize = XT_ALIGN(sizeof(struct xt_mcast_target_info)),
.me = THIS_MODULE,
};
static int __init mcast_tg_init(void)
{
mgrp_hash = init_table(group_size);
msrc_hash = init_table(source_size);
get_random_bytes(&jhash_rnd, sizeof(jhash_rnd));
return xt_register_target(&mcast_tg_reg);
}
static void __exit mcast_tg_exit(void)
{
delete_table(mgrp_hash, group_size);
delete_table(msrc_hash, source_size);
xt_unregister_target(&mcast_tg_reg);
}
module_init(mcast_tg_init);
module\_exit(mcast\_tg\_exit);
MODULE\_LICENSE("GPL");
MODULE\_ALIAS("ipt\_MCAST");
MODULE\_DESCRIPTION("Xtables: Mcast packet filter");
MODULE\_AUTHOR("Bewketu Tadilo<xxxx@xxxx.edu.au>");
\end{verbatim}
\subsection{libxt\_MCAST.c (for userland interaction to set values).}
\begin{verbatim}
#include <getopt.h>
#include <stdio.h>
#include <string.h>
#include <xtables.h>
#include <linux/netfilter/x_tables.h>
#include "xt_MCAST.h"
static void mcast_tg_help(void)
{
printf("MCAST target options:\n"
" --multicast-source-size value expected number of multicast sources\n"
" --multicast-group-size value expected number of multicast groups \n");
}
enum {
MCAST_OPT_MSRC_SIZE,
MCAST_OPT_MGRP_SIZE,
};
static const struct option mcast_opts[] = {
{"multicast-source-size", 1, NULL, MCAST_OPT_MSRC_SIZE},
{"multicast-group-size", 1, NULL, MCAST_OPT_MGRP_SIZE},
{.name = NULL},
};
static int mcast_tg_parse(int c, char **argv, int invert,
unsigned int *flags, const void *entry,
struct xt_entry_target **target)
{
struct xt_mcast_target_info *info = (void *) (*target)->data;
unsigned int num;
switch (c) {
case MCAST_OPT_MSRC_SIZE:
if (*flags & (1 << c)) {
xtables_error(PARAMETER_PROBLEM,
"MCAST: can't specify ‘--multicast-source-size' twice");
}
if (!xtables_strtoui(optarg, NULL, &num, 1, UINT32_MAX)) {
xtables_error(PARAMETER_PROBLEM,
"Unable to parse ‘%s' in "
"‘--mulitcast-souce-size'", optarg);
}
info->msrc_size = num;
*flags |= 1 << c;
break;
case MCAST_OPT_MGRP_SIZE:
if (*flags & (1 << c))
xtables_error(PARAMETER_PROBLEM,
"MCAST: can only specify ‘--multicast-group-size' once");
if (!xtables_strtoui(optarg, NULL, &num, 1, UINT32_MAX)) {
xtables_error(PARAMETER_PROBLEM,
"Unable to parse ‘%s' in "
"‘--mulitcast-souce-size'", optarg);
}
info->mgrp_size = num;
*flags |= 1 << c;
break;
default:
return 0;
}
return 1;
}
static void mcast_tg_check(unsigned int flags)
{
}
static struct xtables_target mcast_tg_reg = {
.version = XTABLES_VERSION,
.name = "MCAST",
.revision = 0,
.family = PF_INET,
.size = XT_ALIGN(sizeof(struct xt_mcast_target_info)),
.userspacesize = XT_ALIGN(sizeof(struct xt_mcast_target_info)),
.help = mcast_tg_help,
.parse = mcast_tg_parse,
.final_check = mcast_tg_check,
};
static __attribute__ ((constructor))
void mcast_tg_ldr(void)
{
xtables_register_target(&mcast_tg_reg);
}
\end{verbatim}
\subsection{libxt\_MCAST.c (for userland interaction to set values).}
\subsection{Scripts.}
\subsubsection{Makefile.}
\begin{verbatim}
top_srcdir := ..
srcdir := .
prefix := /usr/
exec_prefix := ${prefix}
libdir := /usr/lib
libexecdir := ${exec_prefix}/libexec
xtlibdir := ${libexecdir}/xtables
xtables_CFLAGS :=
CC := gcc
CCLD := ${CC}
LCFLAGS
:= -g -O2
LDFLAGS :=
regular_CFLAGS := -D_LARGEFILE_SOURCE=1 -D_LARGE_FILES -D_FILE_OFFSET_BITS=64
-D_REENTRANT -Wall -Waggregate-return -Wmissing-declarations
-Wmissing-prototypes -Wredundant-decls -Wshadow -Wstrict-prototypes
-Winline -pipe -DXTABLES_LIBDIR=\"${xtlibdir}\"
kinclude_CFLAGS :=
AM_CFLAGS
-I /lib/modules/2.6.25.20-0.1-default/build/include
:= ${regular_CFLAGS} -I${top_srcdir}/include ${xtables_CFLAGS}
${kinclude_CFLAGS}
AM_DEPFLAGS
= -Wp,-MMD,$(@D)/.$(@F).d,-MT,$@
obj-m += compat_xtables.o xt_MCAST.o
all: lib
make -C /lib/modules/‘uname -r‘/build M=‘pwd‘
clean:
make -C /lib/modules/‘uname -r‘/build M=‘pwd‘ clean
rm -rf libxt_MCAST.so
install:
cp *.so
$(xtlibdir)
cp *.ko /lib/modules/‘uname -r‘/extra
lib:
libxt_MCAST.so
lib%.so: lib%.oo
${CCLD} ${AM_LDFLAGS} -shared ${LDFLAGS} -o $@ $<;
lib%.oo: ${srcdir}/lib%.c
${CC} ${AM_DEPFLAGS} ${AM_CFLAGS} -D_INIT=lib$*_init -DPIC -fPIC
${LCFLAGS} -o $@ -c $<;
\end{verbatim}
\subsubsection{iptables rule (test.sh).}
\begin{verbatim}
#!/bin/sh
IPT=/usr/sbin/iptables
modprobe x_tables
$IPT -F
rmmod
xt_MCAST
rmmod compat_xtables.ko
insmod compat_xtables.ko
insmod xt_MCAST.ko
$IPT -I OUTPUT -p udp --dport 4444 -j MCAST
#snoop ONLY igmp we don't care about outgoing Multicast traffic
$IPT -I OUTPUT -p igmp -j MCAST
$IPT -I INPUT -d 224.0.0.0/4 ! -p igmp -j MCAST
#apply policy on incoming mcast traffic which is IGMP (it could be proxied)
\end{verbatim} 
\chapter{Userland Simulator}
To get the following work one needs /linux/list.h which can be dowloaded
(by easy google search) or in /usr/src/linux/include/linux/list.h if one has source
installed.
• Remove all the prefetch statements
• Remove all the kernel-specific constants
• put containerof implementation in it as follows and can include it in user
(as opposed to kenel) application by #include :
#define container_of(ptr, type, member) ({ \
const typeof( ((type *)0)->member ) *__mptr = (ptr);
\
(type *)( (char *)__mptr - offsetof(type,member) );})
#define offsetof(TYPE, MEMBER) ((size_t) &((TYPE *)0)->MEMBER)
\section{Header File}
\begin{verbatim}
#ifndef _MTEST_H
#define _MTEST_H
#include <stdint.h>
struct xt_mcast {
struct hlist_node node;
__be32 ip;
unsigned long timeout;
//unsigned long
expire; //this is in jiffies (allowed per group)
};
struct mtable {
struct hlist_head members[0];
};
#define time_after(a,b) \
((long)(b) - (long)(a) < 0)
#define time_before(a,b) time_after(b,a)
static unsigned int
init_new_entry(struct xt_mcast *entry, __be32 ip, uint32_t h,
struct mtable *t, uint16_t expire);
static inline uint32_t mcast_hash(__be32 name, unsigned int size);
static struct hlist_node *lookup(__be32 hkey, __be32 ip, struct mtable *t);
static struct hlist_node *get_headnode(struct mtable *t, uint32_t h);
static unsigned int igmp_report(__be32 cli, __be32 grp);
static unsigned int mcast_search(__be32 msrs_ip, __be32 mgrp_ip);
static unsigned igmp_leave(__be32 cli, __be32 grp);
static inline struct mtable *init_table(uint32_t nmembers);
static void delete_list(struct hlist_head *head);
static inline void delete_table(struct mtable *t, uint32_t nsize);
static unsigned int unicast_handler(__be32, __be32);
#endif /*_MTEST_H*/
\end{verbatim}
\section{Main.c}
\begin{verbatim}
#include <stdio.h>
#include <stdbool.h>
#include <time.h>
#include <linux/types.h>
#include <stdlib.h>
#include <netinet/ip.h>
#include <netinet/in.h>
#include <netinet/ip6.h>
#include <getopt.h>
#include "list.h"
#include "mtest.h"
#include <linux/netfilter.h>
#include "jhash.h"
#include <sys/time.h>
#define MSRC_SIZE 20 //src->cli_head_nodes
#define MGRP_SIZE 20 //grp->cli_head_nodes
#define MGRP_EXPIRE
120
//2-minutes about the same timetime router will issue Group Query.
#define MCAST_GRP_ID
10
#define ARRAR_SIZE(array) (sizeof(array)/sizeof(array[0]))
static void print_hash_list(struct hlist_head head);
void print_usage(FILE * stream, char **argv);
static inline void simulate(int, char **);
static struct mtable *msrc_hash;
//client list indexed by hash(src)->cli_ip
static struct mtable *mgrp_hash;
//client list indexed by hash(gid)->cli_ip
static unsigned int source_size = MSRC_SIZE;
static unsigned int group_size = MGRP_SIZE;
static unsigned int source_expire = 2 * 60 * 60;
//2-hours
static unsigned int *rand_src, *rand_grp;
static unsigned int *rand_cli;
static struct timeval now;
static u_int32_t jhash_rnd;
// all insert functions are f(keyip,cli)
format
int main(int argc, char **argv)
{
int i;
if (argc < 2)
print_usage(stderr, argv);
jhash_rnd = time(NULL);
mgrp_hash = init_table(group_size);
msrc_hash = init_table(source_size);
simulate(argc, argv);
// for(i = 0; i < 10; i++)
//
igmp_report(i,MCAST_GRP_ID);
//print_hash_list(mgrp_hash->members[mcast_hash(MCAST_GRP_ID,group_size)]);
//unicast_handler(10,12);
//mcast_search(MCAST_GRP_ID,9);
delete_table(mgrp_hash, group_size);
delete_table(msrc_hash, source_size);
return 0;
}
static struct option long_opt[] = {
{"ngrp", 1, 0, 'g'},
{"ncli", 1, 0, 'c'},
{"nsrc", 1, 0, 's'},
{"percentag", 1, 0, 'r'},
{NULL, 0, NULL, 0}
};
static char *short_opt = "g:c:s:r:";
static inline void simulate(int argc, char **argv)
{
//15-groups
//15-clients
int i;
srand(time(NULL));
int c, sizec, sizes, sizeg, per;
long mtime, seconds, useconds;
struct timeval start;
struct timeval end;
long average = 0;
while (1) {
c = getopt_long(argc, argv, short_opt, long_opt, NULL);
if (c == -1)
break;
switch (c) {
case 'c':{
sizec = atoi(optarg);
rand_cli = malloc(sizec * sizeof(int *));
for (i = 0; i < sizec; i++)
rand_cli[i] = rand();
}
break;
case 's':{
sizes = atoi(optarg);
rand_src = malloc(sizes * sizeof(int *));
for (i = 0; i < sizes; i++)
rand_src[i] = rand();
}
break;
case 'g':{
sizeg = atoi(optarg);
rand_grp = malloc(sizes * sizeof(int *));
for (i = 0; i < sizes; i++)
rand_grp[i] = rand();
}
break;
case 'r':
per = atoi(optarg);
break;
default:
print_usage(stderr, argv);
}
}
for (i = 0; i < sizec; i++) {
igmp_report(rand_cli[i], rand_grp[rand() % sizeg]);
unicast_handler(rand_cli[i], rand_src[rand() % sizes]);
}
//again here now we simulate a random source contacts our network
//here we measure how long does it take
//to block it to let it go with a relation to the our memory
//do 300 random searches printing the raw time it took.
for (i = 0; i < 300; i++) {
gettimeofday(&start, NULL);
mcast_search(rand_src[rand() % sizes], rand_grp[rand() % sizeg]);
gettimeofday(&end, NULL);
//seconds = end.tv_sec - start.tv_sec;
useconds = end.tv_usec - start.tv_usec;
average += useconds;
//
mtime=
(seconds *1000 + useconds/1000.0)+ 0.5;
//printf("Elapsed: %ld us\n",useconds);
}
printf("Averag elapsed %ld useconds\n", average / 300);
}
static void print_hash_list(struct hlist_head head)
// this is array of hashtables
{
struct xt_mcast *entry;
struct hlist_node *pos;
hlist_for_each_entry(entry, pos, &head, node)
printf("ip: %u\n", entry->ip);
}
static unsigned int
init_new_entry(struct xt_mcast *entry, __be32 ip, uint32_t h,
struct mtable *t, uint16_t expire)
{
struct hlist_node *pos, *temp;
struct xt_mcast *temp_entry;
struct xt_mcast *place;
struct hlist_node *ref = get_headnode(t, h);
hlist_for_each_entry_safe(place, pos, temp, &t->members[h], node) {
gettimeofday(&now, NULL);
if (place != NULL && time_after(now.tv_sec, place->timeout)) {
entry = place;
pos = pos->next;
//don't delete pos- just advance it
while (pos != NULL && pos != ref) {
temp_entry = hlist_entry(pos, struct xt_mcast, node);
free(temp_entry);
temp = pos;
pos = NULL; //No pter reshuffle here
pos = temp->next;
}
break;
}
}
if (entry == NULL)
entry = (struct xt_mcast *) malloc(sizeof(struct xt_mcast));
entry->ip = ip;
gettimeofday(&now, NULL);
entry->timeout = now.tv_sec + expire;
hlist_add_head(&entry->node, &t->members[h]);
return NF_ACCEPT;
}
static inline uint32_t mcast_hash(__be32 name, unsigned int size)
{
return jhash_1word(name, jhash_rnd) & (size - 1);
}
static struct hlist_node *lookup(__be32 hkey, __be32 ip, struct mtable *t)
{
struct xt_mcast *entry;
struct hlist_node *n;
struct hlist_node *temp = n;
//use of the mutex should be appropriate here-Noisy Kernel unless new!!
hlist_for_each_entry_safe(entry, n, temp, &t->members[hkey], node)
if (ip == entry->ip)
return n;
return NULL;
}
static struct hlist_node *get_headnode(struct mtable *t, uint32_t h)
{
return t->members[h].first;
}
static unsigned int igmp_report(__be32 cli, __be32 grp)
{
struct xt_mcast *entry = NULL, *head_entry;
//aka head
uint32_t h = mcast_hash(grp, group_size);
struct hlist_node *head_node = get_headnode(mgrp_hash, h);
struct hlist_node *entry_node;
if (head_node == NULL)
return init_new_entry(entry, cli, h, mgrp_hash, MGRP_EXPIRE);
//go fish
entry_node = lookup(h, cli, mgrp_hash);
//here there is grp alrdy but is the client in it.
if (entry_node == NULL) {
return init_new_entry(entry, cli, h, mgrp_hash, MGRP_EXPIRE);
}
//good the client is in the grp.
head_entry = hlist_entry(head_node, struct xt_mcast, node);
entry = hlist_entry(entry_node, struct xt_mcast, node);
gettimeofday(&now, NULL);
entry->timeout = now.tv_sec + MGRP_EXPIRE;
//we need to also update the same entry in the source list
if (entry->ip == head_entry->ip)
return NF_ACCEPT;
//it the head. don't do anything
hlist_del(entry_node); //shuffle and fix
hlist_add_head(&entry->node, &mgrp_hash->members[h]);
//we want this node to be head so we need to del:
//NOTE: when deleting a grp if it the head pointer
printf("IGMP REPORT END\n");
return NF_ACCEPT;
}
static unsigned int mcast_search(__be32 msrc_ip, __be32 mgrp_ip)
{
uint32_t hs = mcast_hash(msrc_ip, source_size);
//h(msrc_ip)->cli_ip
uint32_t hg = mcast_hash(mgrp_ip, group_size);
//h(mgrp_ip)->cli_ip
struct xt_mcast *msrc2cli, *mgrp2cli;
struct hlist_node *n, *n2;
//check if there exists any client(s) that are mapped to msrc_ip
//while looking hash them and check if they have
hlist_for_each_entry(msrc2cli, n, &msrc_hash->members[hs], node) {
hlist_for_each_entry(mgrp2cli, n2, &mgrp_hash->members[hg], node) {
if (msrc2cli->ip == mgrp2cli->ip) { //success
return NF_ACCEPT;
//remember we won't update anything here (
}
}
}
return NF_DROP;
}
static unsigned igmp_leave(__be32 cli, __be32 grp)
{
uint32_t h = mcast_hash(grp, group_size);
struct hlist_node *entry_node;
struct xt_mcast *entry, *head_entry; //aka head
struct hlist_node *head_node = get_headnode(mgrp_hash, h);
/* entry is NULL => "I am leaving this grp"-
yet there is no such grp or doesn't have me
* shouldn't be possible */
if (head_node == NULL)
return NF_ACCEPT;
entry_node = lookup(h, cli, mgrp_hash);
if (entry_node == NULL) //weird
return NF_ACCEPT;
entry = hlist_entry(entry_node, struct xt_mcast, node);
head_entry = hlist_entry(head_node, struct xt_mcast, node);
entry = hlist_entry(entry_node, struct xt_mcast, node);
if (head_entry->ip == entry->ip && entry_node->next != NULL)
//promote the successor to be head
hlist_add_head(entry_node->next, &mgrp_hash->members[h]);
//else it is the last entry
hlist_del(entry_node);
free(entry);
return NF_ACCEPT;
}
static inline struct mtable *init_table(uint32_t nmembers)
{
uint32_t i;
struct mtable *t;
t = malloc(sizeof(*t) + nmembers * sizeof(t->members[0]));
for (i = 0; i < nmembers; i++)
INIT_HLIST_HEAD(&t->members[i]);
return t;
}
static void delete_list(struct hlist_head *head)
{
struct xt_mcast *entry;
struct hlist_node *pos, *temp;
pos = head->first;
while (pos != NULL) {
entry = hlist_entry(pos, struct xt_mcast, node);
temp = pos;
pos = NULL; //No pter reshuffle here
pos = temp->next;
free(entry);
}
}
static inline void delete_table(struct mtable *t, uint32_t nsize)
{
uint32_t i;
if (t == NULL)
return;
for (i = 0; i < nsize; i++)
delete_list(&t->members[i]);
free(t);
}
static unsigned int unicast_handler(__be32 mcli_ip, __be32 msrc_ip)
{
uint32_t hs = mcast_hash(msrc_ip, source_size);
struct hlist_node *mcli_node;
struct xt_mcast *mcli = NULL;
mcli_node = lookup(hs, mcli_ip, msrc_hash);
if (mcli_node == NULL)
return init_new_entry(mcli, mcli_ip, hs, msrc_hash, source_expire);
mcli = hlist_entry(mcli_node, struct xt_mcast, node);
gettimeofday(&now, NULL);
mcli->timeout = now.tv_sec + source_expire;
return NF_ACCEPT;
}
void print_usage(FILE * stream, char **argv)
{
fprintf(stream, "\n" "Usage: %s -c <ncli> " "-s <nsrc> \n", argv[0]);
fprintf(stream,
"\n"
"Opitions:\n"
" -c, --ncli number of clients\n"
" -s, --nsrc expected number of sources\n"
" -r, --percentage multicast connection percentage (from the above)\n"
"\n");
exit(0);
}
\end{verbatim}
\backmatter
\newcommand{\refname}{References}
\renewcommand{\bibname}{References}
%\newcommand{\myreftoc}{\verbatim{\contentsline{section}{\numberline{\secnumdepth}section}}
%\addtocontents{toc}{\reftoc {\thepage}{\refname}}
\refstepcounter{chapter}
\addcontentsline{toc}{chapter}{\bibname}
%\addtocounter{\secnumdepth}{1}
%\chapter
\begin{thebibliography}{20}
\bibitem{Bradst} Brad Cain, Steve Deering, Isidor Kouvelas, Bill Fenner, and Ajit Thyagarajan. Internet
Group Management Protocol, Version 3. Request For Comments (RFC) 3376, October 2002.
\bibitem{Deering} S. Deering. Host Extensions for IP Multicasting. Request For Comments (RFC) 1112, August
1989. http://www.ietf.org/rfc.html. 1
\bibitem{wfenner} W. Fenner. Internet Group Management Protocol, Version 2. Request For Comments (RFC)
2236, November 1997. http://www.ietf.org/rfc.html. 2.2.2
\bibitem{Deeringmld} Steve Deering, Bill Fenner, and Brian Haberman. Multicast Listener Discovery (MLD) for
IPv6. Request For Comments (RFC) 2710, October 1999. http://www.ietf.org/rfc.html.
\bibitem{Deeringacm}S.DEERING and DAVID R. CHERITON, ACM Transactions on Computer Systems, Vol. 8,
No. 2, May 1990,
\bibitem{RadiaRouters}Radia Perlman, Interconnections: Bridges and Routers, Addison-Wesley Professional Com-
puting Series 1999. 2.3
\bibitem{Stevensunix}W.R. Stevens Unix Network Programming,Volume 1, Third Edition, Prentice Hall 2004. 2.3
\bibitem{Sivaraman}S. Li, V. Sivaraman, A. Krumm-Heller, C. Russell, "A Dynamic Stateful Multicast Firewall",
IEEE International Conference on Communications (ICC), Glasgow, Scotland, Jun 2007. 1.3
\bibitem{holbrookssm}
H. Holbrook, B. Cain “Source-Specific Multicast for IP”, draft-ietf-ssm-arch-07, 4 Oct 2005 2.4
\bibitem{ssmap} SSM-Mapping, http://www.cisco.com/en/US/docs/ios/ipmulti/configuration/guide/imc\_ssm\_mapping.pdf
2.5.2.1
\bibitem{paulMos}Paul Judge and Mostafa Ammar, Security Issues and Solutions in Multicast Content Distri-
bution: A Survey, IEEE Network. January/February 2003. 2.5.1
\bibitem{Loriamf} L. Oria. Approaches to Multicast over Firewalls: An Analy- sis. Technical Report HPL-
IRI-1999-004, HP Labs, Aug 1999. http://www.hpl.hp.com/techreports/1999/HPL-IRI-1999-004.html. 2.5.2
\bibitem{IETFMsec} IETF Multicast Security (MSEC) working group. http://www.ietf.org/html.charters/msec-charter.html. 2.5.1
\bibitem{shaiR}Shai Rubin,David Bernstein,and Michael Rodeh,IBM Research Lab in Haifa and Computer Science Department – Technion (Israel Institute of Technology)- 1999 http://pages.cs.wisc.edu/~shai/CC99.pdf 7
\bibitem{paulERCU} Paul E. McKenney. Using RCU in the Linux 2.5 kernel. Linux Journal, 1(114):18–26, October
2003. 7
\bibitem{BobJhash} Bob Jenkins,A hash function for hash Table lookup,http://www.burtleburtle.net/bob/hash/doobs.html
\bibitem{Paulnetfil} Paul Russel
and Harald Welte,“Netfilter Hacking How-to”: http://www.netfilter.org/documentation/HOWTO/netfilter-hacking-HOWTO.txt.
\bibitem{kernipikepract} B. W. Kernighan and R. Pike, The Practice of Programming, Addison-Wesley, 1999. 3.2
\bibitem{UlrichMemor} Ulrich Drepper,What Every Programmer Should Know About Memory, Red Hat, Inc. November 21, 2007.
\end{thebibliography}

