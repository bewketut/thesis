\chapter{Discussion}
\section{Mitigating Unicast Starvation}
The strongest assumption we made as discussed was that of unicast interaction
of sources and clients. Here are some ways that may be able to help mitigate those
problems:
\begin{itemize}
\item If a user/multicast application is not getting its/his traffic, see what ports,
protocols and unicast control message the application use and add those
to the monitoring list for unicast snooping.
\item It is often helpful to monitor web browsing (but it may increase the state)-
it is again up to the Network Manager to decide by adjusting the memory
required and the expire time of sources.
\item If other means do not work , explicit pings to the unicast sources are
always easy fixes.
\end{itemize}
\section{Another Case Against NATs}
NATs are haunted by controversies but we add one more of our own. Imagine
the following scenario: a source from NATted network is a valid and one of clients
tries to contact it. Due to the nature of the network on the side, our module will
assume all sources from the NATed network are valid. The problem is artefactual
and we do not consider it in our design as based on Engineering principles.
\section{Why Access List Security can be Hard Multicast?}
Multicasting has a seemingly very strange semantics. If two processes open the
same port as a multicast socket and then join two different multicast groups then
traffic for both multicast groups is forwarded to either process. This means that
application will get surprising data that they did not ask for. Applications will
have to filter these out in order to work correctly if multiple applications run on
the same system.
The intent was that multicast address socket behavior would be just like unicast
address behavior, with the exception that multicast addresses are interface-specific.
If we add a new unicast address to the machine, and the binding on the socket
doesn't restrict it, you will receive packets from any of the addresses on the machine.
That's what INADDR\_ANY means. Your multicast application may also receive
unicast traffic on that port too.
Some operating systems supporting Multicast have since changed to only sup-
plying multicast traffic to a socket that was selected through multicast join op-
erations. (application will no longer receive traffic from multicast groups that it
did not subscribe to.) That does not totally avoid the problem except violating
the RFC which is explained in the second paragraph. The problem is anyone on
the network, or within multicast routing domain, may reuse both-multicast group
and port- for a different machine and the application will receive them. The other
machine will also receive the traffic destined for the application.
Why did we go to 'unnecessary' detail above? It should be clear the virtualness
of multicast address makes it easy for anyone to get others people data and fill
garbage on the group address. In security point of view, there is a breach if someone
has access to the data even if they cannot decrypt it.

