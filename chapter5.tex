\chapter{Experiments and Results}
\begin{quote}
pigram 95. Don't have good ideas if you aren't willing to be re-
sponsible for them.
-Alan J. Perlis
\end{quote}
\section{Basics Functionality Setup}
\subsection{Tools needed.}
\begin{itemize}
\item Linux based system as outlined in README in Appedix A.1.
\item Iptables Netfilter Modules as outlined in the README in AppendixA.1.
\end{itemize}
The easiest way to check the functionality is to start a streaming video on VLC1
server. This is simply done using GUI of the player. Then, start that session in
VLC client.\footnote{This a very good media player and can be found on www.videolan.org} Again this is done using GUI feature of VLC. One can use the shell
scripts given in Appendix to test the blocking and unblocking functionalities.
\section{Resilence Test (Simulation)}
\subsection{Test platform.}
Operating System: OpenSuSE11.0 with Linux kernel version 2.6.25
Architecture: x86\_64 Pentium D
Memory: 2GB (RAM 2DDR)
\subsection{What we are measuring.} The main performance concern as previ-
ously mentioned is the latency. Therefore, what we need to test is how long will a
packet stay in our module before a decision is made to either drop or accept that
packet. For that, a userland simulator is rewritten which is based on the main
kernel module. There was a time and resource constraint to test the module on
real systems.
For the sake of getting pattern on how increasing number of clients, groups and
sources contacted influence, the numbers are made large. This is in contrast with
real networks where hundreds of clients might only be in the network. The source code is provided
in Appendix B.
As can be infered from the source code, we have made the following assumptions:
\begin{itemize}
\item The space is kept at only 20 buckets (array)- again this is for the sake
getting a pattern not because of lack of memory.
\item Every client is participating in at least one random group.
\item Every client contacts some random external host as its source.
\end{itemize}

Then what we try to simulate is basically the searching (at peak time when none
of the groups or sources have expired). It would also be interesting to see what the
effect of expiring time has. We may have to defer that to our recommendation. We
have not made any assumption whether the source is malcious or not. We average
the time it takes for a packet to pass through or be dropped over three hundred
source requests (a source delivering a multicast packet) .We run that again on the
terminal 10 times and average the result.
\subsection{Number of clients as variable.} We keep number of groups at 100
and number of sources 10000
\begin{tabular}{|l| l| l| l| l| l| l|l|l|l|l|}
\hline
Clients& 200 & 500 & 600 & 700 & 800 & 900 & 1800 & 3000 & 4000 & 10,000\\
\hline
Time (\mu s)& 2 & 3 & 4 & 5 & 8 & 9 & 24 & 45 & 60 & 150\\
\hline
\caption{Number of clients vs raw processing time}
\end{tabular}\\


25
Clients effect
20
15
10
5
0
500
1,000
1,500
clients
Figure 5.1. Effect of Number of Clients on latency
\subsection{Number of groups as variable.} We keep the number of clients at
1000 and the sources at 10000. The assumption being one client contacts to ten
possible unicast addresses at a time (e.g web browsing different sites.)\\
\begin{tabular}{|l| l| l| l| l| l| l|l|l|}
\hline
Groups & 10 & 20 & 50 & 100 & 200 & 500 & 1000 & 2000\\
\hline
Time (\mu s)& 5 & 8 & 10 & 10 & 11 & 11 & 10 & 7\\
\hline
\caption{Number of groups vs raw processing time}
\end{tabular}\\
Groups effect
10
8
6
0
500
1,000
groups
1,500
2,000
Figure 5.2. Effect of Number of Groups on latency
\section{Number of sources as variable.} We keep the number of groups 100
and clients at 1000.
sources effect
\begin{tabular}{|l| l| l| l| l| l| l|l|l|l|}
\hline
Sources &1 & 10 & 20 & 50 & 100 & 200 & 500 & 1000 & 10000\\
\hline
Time (\mu s)& 5 & 6 & 10 & 8 & 11 & 10 & 10 & 7 & 6\\
\hline
\end{tabular}\\

10
8
6
0
0.2
0.4
0.6
sources
0.8
1
·104
Figure 5.3. Effect of Number of Sources on latency
\section{Results Interpretation}
The results shows the module's dependence on internal clients number- as would
be expected. The more clients we have the more states we have to keep. In other
words, it is our clients that cause the state information creation by initiating those
contacts. If we have more clients, they are likely to contact more groups and group
sources.
More interestingly though, the more groups we have outside, the more likely
our clients will contact different groups. This causes a faster hash access to the
clients. Remember we use the group and source IP addresses as hash keys to store
the clients and then do a cross check. If these values become diverse, which is
what happens when we increase their domain, their hash value is likely to differ
too and thus causing less collisions. This is what is apparent on the the latter
two graphs and on my tests. This aligns perfectly with the design challenge of
the dynamic multicast firewall, which does not know what the source IP or group
IP are before and considers them random. Be warned though, this assumption of
randomness may not be true in practice. And also, the actual network load (not
our management) maybe unbearable when so many sources and groups exist.
Lastly, even if the results put are in the order of $\mu$s, we need to bear in mind
multimedia applications (e.g IPTV) that mulitcasting is used for are in the order of
M bps. The result can be that we are not processsing as fast or faster than the bitrate
which is undesired for the application. We would be introducing artifact bandwidth
throttling by our module. For instance, if we are watching real-time video, the
implication is we no longer can watch it properly not because of lack of bandwidth
but the fact that the module has clogged it. Therefore, the machine's raw processing
power and other features are also important in these type of situations and need
considerations.

