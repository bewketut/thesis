\chapter{Detailed Design}
\begin{quote}
Make everything as simple as possible, but not simpler.
-Albert Einstein\end{quote}
\section{Design Dilemmas}
We start with something we have implicitly deferred for a while. This is the
case of IGMP leave messages. They have been left out from the code flow chart of
the previous Chapter intentionally. Leave message can be deceptive in the spirit
that we discussed as comeback leave & genuine leave earlier. The design dilemma
posed by that is: do we go to that particular group hash and remove the client
entry from there or simply ignore them?. If we remove the client to just discover
it comes back would be inefficient. IGMP leave messages are, of course, good for
routing scalability. In our case though, they can destroy the cache like behaviour
we worked hard for by removing the client prematurely. Even though we included
that implementation in our module, simplicity and less work bears against it.
Another dilemma is ’how long should a source be valid for?’. That is once it had
unicast interaction with one of our clients. There is no good answer. In the module
we passed that question to the Network Manager to decide (using the userland
Iptables program s/he can set the value on command line). Imagine our IPTV user
again and a hypothetical scenario. S/he switches on the box and during booting
the box contacts the content providers IP address (maybe during that action the
filter notices the contacted IP address). The user is passively watching a movie
from then on and no unicast interaction occurs. It would be undesirable to drop
the packets while the user is watching.
\section{Where is the Module and Why?}
As can be seen on Figure 4.1, our Filter is located right after the switch. This
mainly due to physical reason of getting the whole traffic in one interface. That is,
the switch is where the traffic is multiplexed and demultiplexed and as such it is
the converging point of outgoing traffic. This is called Bridge firewall- Bridgewall-
which is also powerful as it can also do layer 2 filtering (we can, for instance, deny
access to a user by MAC address-IP spoofing impossible- who is contacting bad IP
addresses).
Figure 4.1. Module location
\section{The Memory Management Paradigm}
One performance issue that dogged a previous student design solution was a
periodic ’call back’ function that goes and cleans entries when a timer expires. And
later on, goes to ask to allocate new memory when new entries appear. This can
keep the processor quite busy with loads and stores to and from main memory
while also destroying possible localities maintained by cache1. Memory allocation
is also an expensive process. And when we request for more memory every now
and then, it leads to O(n) linked list complexity (one fragmented memory points to
next as standard linked list structure) to access what we want. This is restatement
of memory ’fragmentation’ problem. There are smart ways of allocating- what the
’slab allocater’ does in the Linux kernel. It is not in the document scope to discuss
further on this matter.We urge the reader for some faith that less call to any of
\_malloc family functions is better. If not convinced we refer for discussion memory
fragmentation and all other memory related stuff to [19].
The fundamental basis behind the design the algorithm about to be discussed
is:
• Asking kernel’s memory management system (the slab allocator in Linux)
for a new block of memory right after freeing one is to be avoided. Instead
1\footnote{This can cause the familiar computer ’crunching’ sound- when programs that (ab)use the memory
too much are running.}
we want to overwrite that with a new entry. This is in the same sense of
’one will not break a plate after eating on it and go out to buy new one
when s/he want to eat again’.
On further resolution the above statement, we can ask when is the right time to
free it and how are we going to decide that memory is not needed anymore? As we
will see in a bit the right time is when we go to that bucket location and we free it
if the timer on that node is expired.
The Lazy Cleaner’s Dilemma* (Optional). We introduce a motivation
for our design that we call the lazy cleaner’s dilemma. The cleaner wants to maxi-
mize cleanness and orderliness of stuff s/he uses but minimize cleaning work. The
cleaning precedes the use; the stuff is considered dirty after some time of its use.
We can raise a questions like ’When is it optimal to wash a cup that he used now to
be considered dirty’. It is not unusual to see people have another tea on the same
cup. Examples of non-optimal solutions: periodic cleaning, cleaning stuff s/he is
not likely to use again. I urge you to map this analogies to memory allocation call
optimisation.
\section{The Lazyadd Algorithm (Insert/Delete)}
Let’s begin with the several assumptions
• We have allocated array of n size (take array size of 8 for illustration),
• Initialised (for each bucket-cell) a head pointer to a a linked list node
structure. That is what we call the initialisation step if it referred to
below. The full glory of the data structure can be seen in Appendix
A.2.22.
• Our value IP address vip to be stored and hash key IP address,kip, just
as below:
hash(kip)
kip
vip
One term that we use is dirty node which we define as a node that a timer
associated with it has expired. We can proceed to the steps.
Our algorithm follows roughly these steps to add vip in a bucket located by
hash[kip]:
\subsection{No collision.}
2The way a linked list is implemented (and how one can utilize that) in the Linux kernel is slightly
different from standard Computer Science courses and for us, it is another matter. We refer to the
source code /linux/list.h if one is interested. It safe to assume standard linked list data structure
with next, prev pointers for each structure to understand these algorithms.
(1) Check the bucket pointed by the hash.
kip
hash(kip)
vip
(2) If the head pointer (which is valid during the initialisation step) still points
to NO linked list, create a entry node- head node3.
kip
hash(kip)
NULL
head
vip
(3) This is a case when the head node is still there but it is dirty (i.e the timer
on it has expired).
kip
hash(ip)
head
vip
(4) From one of the two above steps, we have memory. In the first case there
was explicit allocation; as there was no entry before, and in the second
case there was entry but was dirty. In this step, the new entry is added.
kip
hash(ip)
head
vip
Now let’s assume we have had collisions (we have more entries in that bucket),
the list is not dirty up until some middle node- the entries are fresh and have not
expired. We put this general case so the other cases can thought out or inferred.
This leads us to the next section:
\subsection{Collision.}
3\footnote{This will be the “head node”. The head pointer is in the one in the array- small dots. Remember
that the picture covers the head pointers when we show nodes- the nodes are sort of “pinned” on
it}
(1) We have a structure that looks like the following
hash(ip)
kip
head
vip
next
nnex
nnnx
(2) This is a very important step and key part in our algorithm.
(a) Traverse the list until we hit the first expired- we grab its space- and
store the new value.
(b) Delete/clean the list downwards.
kip
hash(ip)
head
vip
next
nnex
nnnx
(3) We restore the property which allows us to do step (2) in two steps
(a) Activate the timer (which no longer makes that node dirty)
kip
hash(ip)
head
vip
next
nnex
(b) This is the last to expire and so make it the head node (the property).
This allows to delete the rest of the list once we find latest expired
node as we did in (2b).
kip
hash(ip)
nnex
vip
head
next
This completes the process of the lazy cleaner insert algorithm.
We have to be careful not to delete the head node at step 2.b if the linked list
structure is circular. This is the case with Linux kernel implementation of the data
structure. Therefore, we need to make sure to hold head node and compare it is
not the node we are at, while dropping4 the expired structures.
\section{Design Tradeoffs}
\subsection{Performance, Memory, Correctness, Design Complexity.} Should
we store the source values that our client is trying to contact? What could it pos-
sibly mean we are not storing those values? Imagine a scenario where a malicious
user’s address maps to an address location which has client (that is due to possible
collision with genuine source). The probability of that happening can be quite sig-
nificant depending the property of the hash function. The module does not check
this but it is slight modification to the data structure including one more ip ad-
dress something I missed. We lose extra storage of memory and one more compare
instruction for the processor in return for correctness.
The performance (which is packet latency at our module) has direct relation
with how many memory cells we allocate to group and client arrays. This is a
simple hashtable tradeoff. We see more of it in Chapter 5.
We can also ask if it is worth the design complexity to, for example, abandon
linked list and use other faster access complexity structures. The answer at this
design stage is no as we have not identified that as bottleneck.
\footnote{As not to cause another ’gravity theory moment’.}

